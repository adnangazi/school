\section*{Scopo dello stage}
L'azienda utilizza un'applicazione non proprietaria per la notarizzazione, ovvero l'autenticazione di documenti. Quando si carica un nuovo documento nel sistema, viene ricavata da esso una chiave poi memorizzata nella Blockchain. 
Quando invece si vuole verificare l'autenticità di un documento precedentemente caricato, esso deve essere un'altra volta aggiunto al sistema, così da potervi calcolare nuovamente una chiave di cui controllare la presenza nella Blockchain. Se il documento sarà stato modificato nel tempo, verrà restituito esito negativo, segnalando il fallimento dell'autenticazione e avvertendo della modifica del documento, poiché la chiave prodotta al momento del caricamento iniziale sarà diversa da quella ricavata al momento dell'autenticazione, in quanto documenti minimamente diversi producono chiavi diverse.
\newline
Si vuole creare una dApp, ovvero applicazione distribuita online, di notarizzazione analoga a quella già in uso dall'azienda, ma di essa proprietaria, in modo da eliminare le dipendenze dai sistemi esterni, riducendo così i costi e semplificando la gestione dell'infrastruttura. Si vuole inoltre arricchire tale dApp con un'interfaccia di testing e monitoraggio, ovvero una piattaforma in grado di eseguire le funzionalità dell'applicazione principale e visualizzarne i risultati al fine testare e monitorare il loro corretto funzionamento. Successivamente, si desidera espandere la dApp di una serie di funzionalità complementari di gestione dei documenti, in grado di permettere di gestire in modo completo i documenti coinvolti nel sistema. 
\newline
Questo progetto serve anche per l'interesse dell'azienda di formare lo studente sugli argomenti del lavoro, in modo non precludere la possibilità di future collaborazioni per proseguire lo sviluppo del sistema. Infatti, L'azienda vuole espandere l'uso della Blockchain in molteplici contesti d'uso. Viene quindi richiesto allo studente di produrre anche una serie di documentazioni in merito allo sviluppo del progetto, in particolare per ogni obiettivo prefissato, facilitando la comprensione e quindi la realizzazione di progetti futuri.