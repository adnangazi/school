\section*{Contenuti formativi previsti}
Lo studente entrerà a contatto con il contesto della Blockchain, sia a livello teorico che a livello pratico per la realizzazione di sistemi che usano la Blockchain, e per l'implementazione di interfacce e funzionalità per esse. 
Per comprenderlo appieno, sarà prima necessario assimilare cognizione e familiarità a livello teorico in diversi domini inerenti la Blockchain e realizzazione di sistemi ad esso annessi. Per potervi invece lavorare, bisognerà precedentemente acquisire conoscenza e dimestichezza a livello prarico con diversi linguaggi di programmazione, librerie e tecnologie.
Inoltre lo studente avrà modo di approfondire e consolidare le proprie conoscenze e capacità in merito allo sviluppo di software, precisamente in progettazione, documentazione, sviluppo, gestione dell'ambiente di lavoro, e verifica e collaudo. Ovviamente, l'intento primario dello stage è soprattutto formare lo studente sul funzionamento di una realtà lavorativa, in modo da potervisi inserire adeguatamente nel mondo del lavoro in futuro.
\newpage